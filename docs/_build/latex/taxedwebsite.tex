%% Generated by Sphinx.
\def\sphinxdocclass{report}
\documentclass[letterpaper,10pt,english]{sphinxmanual}
\ifdefined\pdfpxdimen
   \let\sphinxpxdimen\pdfpxdimen\else\newdimen\sphinxpxdimen
\fi \sphinxpxdimen=.75bp\relax
\ifdefined\pdfimageresolution
    \pdfimageresolution= \numexpr \dimexpr1in\relax/\sphinxpxdimen\relax
\fi
%% let collapsible pdf bookmarks panel have high depth per default
\PassOptionsToPackage{bookmarksdepth=5}{hyperref}

\PassOptionsToPackage{booktabs}{sphinx}
\PassOptionsToPackage{colorrows}{sphinx}

\PassOptionsToPackage{warn}{textcomp}
\usepackage[utf8]{inputenc}
\ifdefined\DeclareUnicodeCharacter
% support both utf8 and utf8x syntaxes
  \ifdefined\DeclareUnicodeCharacterAsOptional
    \def\sphinxDUC#1{\DeclareUnicodeCharacter{"#1}}
  \else
    \let\sphinxDUC\DeclareUnicodeCharacter
  \fi
  \sphinxDUC{00A0}{\nobreakspace}
  \sphinxDUC{2500}{\sphinxunichar{2500}}
  \sphinxDUC{2502}{\sphinxunichar{2502}}
  \sphinxDUC{2514}{\sphinxunichar{2514}}
  \sphinxDUC{251C}{\sphinxunichar{251C}}
  \sphinxDUC{2572}{\textbackslash}
\fi
\usepackage{cmap}
\usepackage[T1]{fontenc}
\usepackage{amsmath,amssymb,amstext}
\usepackage{babel}



\usepackage{tgtermes}
\usepackage{tgheros}
\renewcommand{\ttdefault}{txtt}



\usepackage[Bjarne]{fncychap}
\usepackage{sphinx}

\fvset{fontsize=auto}
\usepackage{geometry}


% Include hyperref last.
\usepackage{hyperref}
% Fix anchor placement for figures with captions.
\usepackage{hypcap}% it must be loaded after hyperref.
% Set up styles of URL: it should be placed after hyperref.
\urlstyle{same}

\addto\captionsenglish{\renewcommand{\contentsname}{Contents:}}

\usepackage{sphinxmessages}
\setcounter{tocdepth}{1}



\title{Taxed website}
\date{Mar 18, 2023}
\release{1.1.0}
\author{Group T2}
\newcommand{\sphinxlogo}{\vbox{}}
\renewcommand{\releasename}{Release}
\makeindex
\begin{document}

\ifdefined\shorthandoff
  \ifnum\catcode`\=\string=\active\shorthandoff{=}\fi
  \ifnum\catcode`\"=\active\shorthandoff{"}\fi
\fi

\pagestyle{empty}
\sphinxmaketitle
\pagestyle{plain}
\sphinxtableofcontents
\pagestyle{normal}
\phantomsection\label{\detokenize{index::doc}}


\sphinxstepscope


\chapter{Installation guide for testing purpose}
\label{\detokenize{pages/installation-testing:installation-guide-for-testing-purpose}}\label{\detokenize{pages/installation-testing::doc}}
\sphinxAtStartPar
In this section, you will be guided on how to download the prototype, create virtual environment, install dependencies and run the website.
If you have any problem, please feel free to contact me via email \sphinxhref{mailto:hien.nguyen@edu.turkuamk.fi}{hien.nguyen@edu.turkuamk.fi} or WhatsApp +358 46 8404770


\section{System requirements}
\label{\detokenize{pages/installation-testing:system-requirements}}
\sphinxAtStartPar
Python version 3

\sphinxAtStartPar
If you have not installed, please visit to \sphinxurl{https://www.python.org/} and comeback once it is installed.


\section{Cloning the repository}
\label{\detokenize{pages/installation-testing:cloning-the-repository}}\begin{enumerate}
\sphinxsetlistlabels{\arabic}{enumi}{enumii}{}{.}%
\item {} 
\sphinxAtStartPar
Clone the repository using the command below:

\begin{sphinxVerbatim}[commandchars=\\\{\}]
\PYGZdl{} git clone https://github.com/hienng9/taxed.git
\end{sphinxVerbatim}

\end{enumerate}

\sphinxAtStartPar
Or download the project directly from \sphinxurl{https://github.com/hienng9/taxed} and extract the zip file to any location that you wish.

\noindent\sphinxincludegraphics{{download-git}.png}
\begin{enumerate}
\sphinxsetlistlabels{\arabic}{enumi}{enumii}{}{.}%
\setcounter{enumi}{1}
\item {} 
\sphinxAtStartPar
In case you use git to clone the project, continue by moving into the directory where we have the project files:

\begin{sphinxVerbatim}[commandchars=\\\{\}]
\PYGZdl{} cd taxed
\end{sphinxVerbatim}

\end{enumerate}

\sphinxAtStartPar
In case you download the project, then open your terminal to change to the directory where the project is located.
From this point on, we will be using the terminal.


\section{Creating a virtual environment}
\label{\detokenize{pages/installation-testing:creating-a-virtual-environment}}
\sphinxAtStartPar
Let’s install virtualenv first if you have not already:

\begin{sphinxVerbatim}[commandchars=\\\{\}]
\PYGZdl{} pip install virtualenv
\end{sphinxVerbatim}

\sphinxAtStartPar
Then we create our virtual environment:

\begin{sphinxVerbatim}[commandchars=\\\{\}]
\PYGZdl{} virtualenv envname
\end{sphinxVerbatim}

\sphinxAtStartPar
Activate the virtual environment using either:

\begin{sphinxVerbatim}[commandchars=\\\{\}]
\PYGZdl{} envname/scripts/activate
\end{sphinxVerbatim}

\sphinxAtStartPar
or:

\begin{sphinxVerbatim}[commandchars=\\\{\}]
\PYGZdl{} source envname/bin/activate
\end{sphinxVerbatim}

\sphinxAtStartPar
Install the requirements:

\begin{sphinxVerbatim}[commandchars=\\\{\}]
\PYGZdl{} pip install \PYGZhy{}r requirements.txt
\end{sphinxVerbatim}


\section{Running the App}
\label{\detokenize{pages/installation-testing:running-the-app}}
\sphinxAtStartPar
To run the App, in the same directory, open one terminal:

\begin{sphinxVerbatim}[commandchars=\\\{\}]
\PYGZdl{} python \PYGZhy{}m celery \PYGZhy{}A taxedwebsite worker \PYGZhy{}l info
\end{sphinxVerbatim}

\sphinxAtStartPar
open another terminal, run the following:

\begin{sphinxVerbatim}[commandchars=\\\{\}]
\PYGZdl{} python manage.py runserver
\end{sphinxVerbatim}

\sphinxAtStartPar
Then, the development server will be started at \sphinxurl{http://127.0.0.1:8000/}

\sphinxstepscope


\chapter{How to navigate through the website}
\label{\detokenize{pages/navigation-bar:how-to-navigate-through-the-website}}\label{\detokenize{pages/navigation-bar::doc}}
\sphinxAtStartPar
In this section, you will know more about navigating in the website.

\sphinxAtStartPar
If you are not logged in, then all you see is the front page tab, salary calculator tab, contact tab and login logo.

\noindent\sphinxincludegraphics{{navbar-notloggedin}.png}
\begin{enumerate}
\sphinxsetlistlabels{\arabic}{enumi}{enumii}{}{.}%
\item {} 
\sphinxAtStartPar
Click on “front page” or the logo image  will lead you to the first page of the website where you can find a lot information about the company, products and services. It is right now still under construction.

\item {} 
\sphinxAtStartPar
“Salary calculator” will lead you to the income calculator where you can know how much you earn after taxes and other funds as well as insurance.

\item {} 
\sphinxAtStartPar
Click on Contact tab if you have any questions or you want to contact us.

\item {} 
\sphinxAtStartPar
Click on Login to login into account or register new user.

\item {} 
\sphinxAtStartPar
Click on the image of messages to start having a conversation with website’s chatbot Tuuli.

\item {} 
\sphinxAtStartPar
Search bar where you can search for information on the website.

\end{enumerate}

\sphinxAtStartPar
For logged\sphinxhyphen{}in users, there are information and tabs.

\noindent\sphinxincludegraphics{{navbar-loggedin}.png}

\sphinxAtStartPar
Information in number 1, 2, 3, 5 and 6 are the same as anonymous users. Logged in users have their own page where they can browse their invoices (\#10), create invoice (\#8), see past invoices (\#9) and check recent activities on the website.

\sphinxAtStartPar
Logged in users can update their information in settings and log out.

\noindent\sphinxincludegraphics{{dropdown}.png}

\sphinxstepscope


\chapter{How to use salary calculator}
\label{\detokenize{pages/salary-calculator:how-to-use-salary-calculator}}\label{\detokenize{pages/salary-calculator::doc}}
\sphinxAtStartPar
In this section, you will find information on using the salary calculator.

\sphinxAtStartPar
Head to the nagivation bar and choose the tab salary calculator.

\noindent\sphinxincludegraphics{{navbar-notloggedin}.png}

\sphinxAtStartPar
Fill in the form the following information.
\begin{enumerate}
\sphinxsetlistlabels{\arabic}{enumi}{enumii}{}{.}%
\item {} 
\sphinxAtStartPar
Total amount of invoice. For example, 3000 euros. In case you are paid hourly and want to charge customers 30 hours for 20 euros per hour, then the total amount of invoice will be 3000 euros.

\item {} 
\sphinxAtStartPar
Your widthholding tax percentage. This can be found in your tax card.

\item {} 
\sphinxAtStartPar
Is VAT included in the total amount of invoice?

\item {} 
\sphinxAtStartPar
if Vat is included int the amount invoicing, please choose the vat rates, either 24\%, 10\% or 0\%.

\item {} 
\sphinxAtStartPar
Age of user. This will be used when user is YEL\sphinxhyphen{}reliable.

\item {} 
\sphinxAtStartPar
Are you YEL\sphinxhyphen{}reliable? An entrepreneur are obliged to take out YEL insurance when their yearly income exceeds a certain amount.

\item {} 
\sphinxAtStartPar
The date when you start taking out YEL insurance. This is used to calculate the percentage of YEL insurance fee that you have to pay.

\end{enumerate}

\sphinxAtStartPar
Click Calculate to show the estimated income after taxes and other fees.

\noindent\sphinxincludegraphics{{calculator-after}.png}

\sphinxstepscope


\chapter{How to interact with chat bot}
\label{\detokenize{pages/chatbot:how-to-interact-with-chat-bot}}\label{\detokenize{pages/chatbot::doc}}
\sphinxAtStartPar
In this section, you will be shown how to make the best out of chat support.

\sphinxAtStartPar
Starting by clicking on the messages icon at the bottom right corner.

\noindent\sphinxincludegraphics{{messages-icon}.png}

\sphinxAtStartPar
The chat support box will appear. User can start interact with chat support by sending different messages.
Examples are shown below.

\noindent\sphinxincludegraphics{{chatbot-joke}.png}

\noindent\sphinxincludegraphics{{chatbot-serious}.png}

\sphinxstepscope


\chapter{User registration, Login and Logout}
\label{\detokenize{pages/user-registration:user-registration-login-and-logout}}\label{\detokenize{pages/user-registration::doc}}
\sphinxAtStartPar
In this section, you will find information on how to create a user acount.


\section{Not yet a user?}
\label{\detokenize{pages/user-registration:not-yet-a-user}}
\sphinxAtStartPar
If you have not had an account yet, register by click on the Login on the top right corner.
When a login box appears, click on sign up to start fill in information.

\noindent\sphinxincludegraphics{{signup1}.png}

\sphinxAtStartPar
In the registration form, fill in all the necessary information such as name, username, email and password. After that, click on “register” button.

\noindent\sphinxincludegraphics{{signup2}.png}


\section{Already a user?}
\label{\detokenize{pages/user-registration:already-a-user}}
\sphinxAtStartPar
In case you already had an account,

\noindent\sphinxincludegraphics{{login}.png}

\sphinxAtStartPar
Either you register new user or login, at the final step, your landing page will be as belows:

\noindent\sphinxincludegraphics{{after-login}.png}

\sphinxstepscope


\chapter{How to update user information}
\label{\detokenize{pages/update-user-info:how-to-update-user-information}}\label{\detokenize{pages/update-user-info::doc}}
\sphinxAtStartPar
In this section, you will be guided on how to change profile picture and update bio information.
First, click on the setting in our avartar logo.

\noindent\sphinxincludegraphics{{update-user1}.png}

\sphinxAtStartPar
Second, choose a profile picture from computer, update your name and fill in the bio section. Please note that the bio section is not allowed to be empty at this stage.

\sphinxAtStartPar
At final step, click on update to update information or cancel to go back to previous page.

\sphinxAtStartPar
If you click on update, you will land on the profile page as follows.

\noindent\sphinxincludegraphics{{user-profile}.png}


\chapter{Indices and tables}
\label{\detokenize{index:indices-and-tables}}\begin{itemize}
\item {} 
\sphinxAtStartPar
\DUrole{xref,std,std-ref}{search}

\end{itemize}



\renewcommand{\indexname}{Index}
\printindex
\end{document}